\documentclass[aspectratio=169]{beamer}

%\documentclass[handout]{beamer}
%% To make 4 per page
%\usepackage{pgfpages}
%\mode<handout>{\setbeamercolor{background canvas}{bg=white}}
%\pgfpagesuselayout{4 on 1}[letterpaper,landscape]%,border shrink=5mm]

\input{slides_setup_nonLightBoard_whiteBG.tex}
% To cross reference from other slide files
\zexternaldocument{MATH-4370-7370-slides-03-eigenpairs-similarity}
\zexternaldocument{MATH-4370-7370-slides-06-nonnegative-matrices}

% To have theorems and everything derived prefixed by the slide set
% number
\renewcommand{\thetheorem}{7.\arabic{theorem}}

\title{Essentially nonnegative matrices and M-matrices}
\author{Julien Arino}
\date{Fall 2025}

%%%%%%%%%%%%%%%%%%%%%
%%%%%%%%%%%%%%%%%%%%%
%%%%%%%%%%%%%%%%%%%%%
%%%%%%%%%%%%%%%%%%%%%
%%%%%%%%%%%%%%%%%%%%%
%%%%%%%%%%%%%%%%%%%%%
%%%%%%%%%%%%%%%%%%%%%
%%%%%%%%%%%%%%%%%%%%%
\begin{document}

% The title page
\begin{frame}[noframenumbering,plain]
  \begin{tikzpicture}[remember picture,overlay]
    \node[above right,inner sep=0pt] at (current page.south west)
    {
        \includegraphics[width=\paperwidth]{title-page-picture.png}
    };
\end{tikzpicture}
	\titlepage
\end{frame}
\addtocounter{page}{-1}
  
  
\begin{frame}{Outline}
	  \tableofcontents[hideallsubsections]
\end{frame}
\addtocounter{page}{-1}


%%%%%%%%%%%%%%%%%%%%%%%%%%%
%%%%%%%%%%%%%%%%%%%%%%%%%%%
%%%%%%%%%%%%%%%%%%%%%%%%%%%
%%%%%%%%%%%%%%%%%%%%%%%%%%%
\section{Essentially nonnegative matrices}

\begin{frame}
The Perron-Frobenius can be applied not only to nonnegative matrices, but also to matrices that are \emph{essentially nonnegative}, in the sense that they are nonnegative except perhaps along the main diagonal
\vfill
\begin{definition}
A matrix $A\in \M_n$ is \defword{essentially nonegative}\index{matrix!essentially nonegative} (or \defword{quasi-positive}\index{matrix!quasi-positive}) if there exist $\alpha \in \IR$ such that $A+\alpha \mathbb{I}\geq 0$
\end{definition}
\vfill
\begin{remark}
    An essentially nonnegative matrix $A$ has non-negative off-diagonal entries. The sign of the diagonal entries is not relevant
\end{remark}
\end{frame}


\begin{frame}
\begin{remark}
Irreducibility of a matrix is not affected by the nature of its diagonal entries.  Indeed, consider an essentially nonnegative matrix $A$. The existence of a directed path in $G(A)$ does not depend on the existence of ``self-loops''. The same is not true of \emph{primitive} matrices, where the presence of negative entries on the main diagonal has an influence on the values of $A^k$ and thus ultimately, on the capacity to find $k$ such that $A^k>0$
\end{remark}
\end{frame}


\begin{frame}
So we can apply the ``weak'' versions of the Perron-Frobenius Theorem (the imprimitive cases in Theorem~\zref{th:PF_allCases}) to $A+\alpha \mathbb{I}$, which is a nonnegative matrix (potentially irreducible). One important ingredient is a result that was proved as Theorem~\ref{th:spectrum_shift}. Namely, that perturbations of the entire diagonal by the same scalar lead to a shift of the spectrum; this is summarised as
\[
\sigma(A+\alpha \mathbb{I})=\{ \lambda_1+\alpha, \dots, \lambda_n+\alpha,\quad \lambda_i\in \sigma(A)\}
\]
\end{frame}



\begin{frame}
\begin{definition}[Spectral abscissa]
Let $A\in \M_n$. The \defword{spectral abscissa} of $A$, $s(A)$, is 
\[
s(A)= \max\{\Rel(\lambda), \lambda\in \sigma(A)\}
\]
\end{definition}
\vfill
\begin{theorem}\label{th:PF_essentially_nonneg}
Let $A\in\M_n(\IR)$ be essentially nonnegative. Then $s(A)$ is an eigenvalue of $A$ and is associated to a nonnegative eigenvector. If, additionally, $A$ is irreducible, then $s(A)$ is simple and is associated to a positive eigenvector
\end{theorem}
\end{frame}



%%%%%%%%%%%%%%%%%%%%%%%
%%%%%%%%%%%%%%%%%%%%%%%
%%%%%%%%%%%%%%%%%%%%%%%
%%%%%%%%%%%%%%%%%%%%%%%
\section{Z-matrices}
\label{sec:Z_matrices}

\begin{frame}
\begin{definition}
A matrix is of class $Z_n$ if it is in $\M_n(\IR)$ and such that $a_{i,j}\leq 0$, $i \neq j$, $i,j=1,\ldots,n$
\[
    Z_n=\{A\in \M_n: a_{i,j}\leq 0, i\neq j\}
\]
We also say that $A\in Z_n$ has the \defword{$Z$-sign pattern}
\end{definition}
\end{frame}


\begin{frame}
\begin{theorem}[\cite{Fiedler2008}]
Let $A\in Z_n$. TFAE
\begin{enumerate}
    \item There is a nonnegative vector $x$ such that $Ax>0$
    \item There is a positive vector $x$ such that $Ax>0$
    \item There is a diagonal matrix $\diag(D)>0$ such that the entries in $AD=[w_{ik}]$ are such that 
    \[w_{ii}> \sum\limits_{k\neq i} | w_{ik}| \forall i\]
    \item For any $B\in Z_n$ such that $A\geq A$, then $B$ is nonsingular 
    \item Every real eigenvalue of any principal submatrix of $A$ is positive.
    \item All principal minors of $A$ are positive
\end{enumerate}
\end{theorem}
\end{frame}

\begin{frame}
\addtocounter{theorem}{-1}
\begin{theorem}[Continued]
\begin{enumerate}
    \setcounter{enumi}{6}
    \item For all $k= 1, \dots, n$, the sum of all principal minors is positive
    \item Every real eigenvalue of $A$ is positive
    \item There exists a matrix $C\geq 0$ and a number $k > \rho(A)$ such that $A= k\mathbb{I}-C$
    \item There exists a splitting $A=P-Q$ of the matrix $A$ such that $P^{-1}\geq 0$, $Q\geq 0$, and $\rho(P^{-1}Q<1)$
    \item $A$ is nonsingular and $A^{-1}\geq 0$
    \item ...
    \item[18] The real part of any eigenvalue of $A$ is positive
\end{enumerate}
\end{theorem}
\end{frame}


\begin{frame}
\textbf{Notation}: $A\in Z_n$ such that any (and therefore all) of these properties holds is a matrix of class $K$ (or a nonsingular $M$-matrix). \\
\vfill
\begin{theorem}
Let $A\in Z= \bigcap\limits_{i=1, \dots}Z_n$ be symmetric. Then $A\in K$ if and only if $A$ is positive define.
\end{theorem}
\end{frame}



%%%%%%%%%%%%%%%%%%%%%%%
%%%%%%%%%%%%%%%%%%%%%%%
%%%%%%%%%%%%%%%%%%%%%%%
%%%%%%%%%%%%%%%%%%%%%%%
\section{Class $K_0$}
\begin{frame}
\begin{theorem}
Let $A\in Z_n$. TFAE
\begin{enumerate}
    \item $A+\varepsilon \II\in K$ for all $\varepsilon>0$
    \item Every real eigenvalue of a principal submatrix of $A$ is nonnegative
    \item All principal minors of $A$ are nonnegative
    \item The sum of all principal minors of order $k= 1, \dots, n$ is nonnegative
    \item Every real eigenvaue of $A$ is nonegative
    \item There exists $C\geq 0$ and $k\geq \rho(C)$ such that $A= k \mathbb{I}-C$
    \item Every eigenvalue of $A$ has nonnegative real part
\end{enumerate}
$A\in Z_n$ such that any of these properties holds is a matrix of class $K_0$
\end{theorem}
\vfill
\begin{theorem}
Let $A\in Z_n$. Assume $A\in K_0$. Then $A\in K$ $\iff$ $A$ nonsingular
\end{theorem}
\end{frame}

%%%%%%%%%%%%%%%%%%%%%%%
%%%%%%%%%%%%%%%%%%%%%%%
%%%%%%%%%%%%%%%%%%%%%%%
%%%%%%%%%%%%%%%%%%%%%%%
\section{M-matrices}



\begin{frame}
\begin{definition}[Signature matrix]
 A \defword{signature matrix} is is a diagonal matrix $S$ with diagonal entries $\pm 1$
\end{definition}   
\end{frame}



\begin{frame}
\begin{theorem}[\cite{BermanPlemmons1994}]
Let $A\in\M_n$. Then for each fixed letter $\C$ representing one of the following conditions, conditions $\C_i$ are equivalent for each $i$. Moreover, letting $\C$ then represent any of the equivalent conditions $\C_i$, the following implication tree holds:
\begin{center}
	\includegraphics[width=0.3\textwidth]{BermanPlemmons_fig}
\end{center}
If $A\in Z_n$, each of the following conditions is equivalent to the statement ``$A$ is a nonsingular M-matrix''
\end{theorem}
\end{frame}

\begin{frame}
\addtocounter{theorem}{-1}
\begin{theorem}[Continued]
\begin{enumerate}
\item[($A_1$)] All the principal minors of $A$ are positive
\item[($A_2$)] Every real eigenvalue of each principal submatrix of $A$ is positive
\item[($A_3$)] For each $\bx\neq\b0$ there exists a positive diagonal matrix $D$ such that
\[
\bx^TAD\bx>0
\]
\item[($A_4$)] For each $\bx\neq\b0$ there exists a nonnegative diagonal matrix $D$ such that
\[
\bx^TAD\bx>0
\]
\item[($A_5$)] $A$ does not reverse the sign of any vector; that is, if $\bx\neq\b0$ and $\by=A\bx$, then for some subscript $i$, $x_iy_i>0$
\item[($A_6$)] For each signature matrix $S$, there exists an $\bx\gg\b0$ such that
\[
SAS\bx\gg\b0
\]
\end{enumerate}
\end{theorem}
\end{frame}

\begin{frame}
\addtocounter{theorem}{-1}
\begin{theorem}[Continued]
\begin{enumerate}
\item[($B_7$)] The sum of all the $k\times k$ principal minors of $A$ is positive for $k=1,\ldots,n$
\item[($C_8$)] $A$ is nonsingular and all the principal minors of $A$ are nonnegative
\item[($C_9$)] $A$ is nonsingular and every real eigenvalue of each principal submatrix of $A$ is nonnegative
\item[($C_{10}$)] $A$ is nonsingular and $A+D$ is nonsingular for each positive diagonal matrix $D$
\item[($C_{11}$)] $A+D$ is nonsingular for each nonnegative diagonal matrix $D$
\item[($C_{12}$)] $A$ is nonsingular and for each $\bx\neq\b0$ there exists a nonnegative diagonal matrix $D$ such that
\[
\bx^TD\bx\neq 0\quad\textrm{and}\quad \bx^TAD\bx>0
\]
\item[($C_{13}$)] $A$ is nonsingular and if $\bx\neq\b0$ and $\by=A\bx$, then for some subscript $i$, $x_i\neq 0$ and $x_iy_i\geq 0$.
\item[($C_{14}$)] $A$ is nonsingular and for each signature matrix $S$ there exists a vector $\bx>\b0$ such that
\[
SAS\bx\geq\b0
\]
\end{enumerate}
\end{theorem}
\end{frame}

\begin{frame}
\addtocounter{theorem}{-1}
\begin{theorem}[Continued]
\begin{enumerate}
\item[($D_{15}$)] $A+\alpha\II$ is nonsingular for each $\alpha\geq 0$
\item[($D_{16}$)] Every real eigenvalue of $A$ is positive
\item[($E_{17}$)] All the leading principal minors of $A$ are positive
\item[($E_{18}$)] There exists lower and upper triangular matrices $L$ and $U$, respectively, with positive diagonals such that
\[
A=LU
\]
\item[($F_{19}$)] There exists a permutation matrix $P$ such that $PAP^T$ satisfies ($E_{17}$) or ($E_{18}$)
\end{enumerate}
\end{theorem}
\end{frame}

\begin{frame}
\addtocounter{theorem}{-1}
\begin{theorem}[Continued]
\begin{enumerate}
\item[($G_{20}$)] $A$ is \defword{positive stable}; that is, the real part of each eigenvalue of $A$ is positive
\item[($G_{21}$)] There exists a symmetric positive definite matrix $W$ such that
\[
AW+WA^T
\]
is positive definite.
\item[($G_{22}$)] $A+\II$ is nonsingular and
\[
G=(A+\II)^{-1}(A-\II)
\]
is convergent
\end{enumerate}
\end{theorem}
\end{frame}

\begin{frame}
\addtocounter{theorem}{-1}
\begin{theorem}[Continued]
\begin{enumerate}
\item[($G_{23}$)] $A+\II$ is nonsingular and for 
\[
G=(A+\II)^{-1}(A-\II)
\]
there exists a positive definite matrix $W$ such that
\[
W-G^TWG
\]
is positive definite
\end{enumerate}
\end{theorem}
\end{frame}

\begin{frame}
\addtocounter{theorem}{-1}
\begin{theorem}[Continued]
\begin{enumerate}
\item[($H_{24}$)] There exists a positive diagonal matrix $D$ such that 
\[
AD+DA^T
\]
is positive definite
\item[($H_{25}$)] The exists a positive diagonal matrix $E$ such that for $B=E^{-1}AE$, the matrix
\[
(B+B^T)/2
\]
is positive definite
\item[($H_{26}$)] For each positive semidefinite matrix $Q$, the matrix $QA$ has a positive diagonal element
\end{enumerate}
\end{theorem}
\end{frame}

\begin{frame}
\addtocounter{theorem}{-1}
\begin{theorem}[Continued]
\begin{enumerate}
\item[($I_{27}$)] $A$ is \defword{semipositive}; that is, there exists $\bx\gg\b0$ with $A\bx\gg\b0$
\item[($I_{28}$)] There exists $\bx>\b0$ with $A\bx\gg\b0$
\item[($I_{29}$)] There exists a positive diagonal matrix $D$ such that $AD$ has all positive row sums
\item[($J_{30}$)] There exists $\bx\gg\b0$ with $A\bx>\b0$ and
\[
\sum_{j=1}^n a_{ij}x_j>0,\quad i=1,\ldots,n
\]
\item[($K_{31}$)] There exists a permutation matrix $P$ such that $PAP^T$ satisfies ($J_{30}$)
\end{enumerate}
\end{theorem}
\end{frame}

\begin{frame}
\addtocounter{theorem}{-1}
\begin{theorem}[Continued]
\begin{enumerate}
\item[($L_{32}$)] There exists $\bx\gg\b0$ with $\by=A\bx>\b0$ such that if $y_{i_0}=0$, then there exists a sequence of indices $i_1,\ldots,i_r$ with $a_{i_{j-1}i_j}\neq 0$, $j=1,\ldots,r$ and with $y_{i_r}\neq 0$
\item[($L_{33}$)] There exists $\bx\gg\b0$ with $\by=A\bx>\b0$ such that the matrix $\hat A=[\hat a_{ij}]$ defined by
\[
\hat a_{ij} =
\begin{cases}
1 & \text{if } a_{ij}\neq 0\text{ or }y_i\neq 0 \\
0 & \text{otherwise}
\end{cases}
\]
is irreducible
\end{enumerate}
\end{theorem}
\end{frame}

\begin{frame}
\addtocounter{theorem}{-1}
\begin{theorem}[Continued]
\begin{enumerate}
\item[($M_{34}$)] There exists $\bx\gg\b0$ such that for each signature matrix $S$
\[
SAS\bx\gg\b0
\]
\item[($M_{35}$)] $A$ has all positive diagonal elements and there exists a positive diagonal matrix $D$ such that $AD$ is \defword{strictly diagonally dominant}; that is
\[
a_{ii}d_i>\sum_{j\neq i}|a_{ij}d_j|,\qquad i=1,\ldots,n
\]
\item[($M_{36}$)] $A$ has all positive diagonal elements and there exists a positive diagonal matrix $E$ such that $E^{-1}AE$ is strictly diagonally dominant
\end{enumerate}
\end{theorem}
\end{frame}

\begin{frame}
\addtocounter{theorem}{-1}
\begin{theorem}[Continued]
\begin{enumerate}
\item[($M_{37}$)] $A$ has all positive diagonal elements and there exists a positive diagonal matrix $D$ such that $AD$ is \defword{lower semistrictly diagonally dominant}; that is,
\[
a_{ii}d_i\geq\sum_{j\neq i}|a_{ij}d_j|,\qquad i=1,\ldots,n
\]
and
\[
a_{ii}d_i>\sum_{j=1}^{i-1}|a_{ij}d_j|,\qquad i=2,\ldots,n.
\]
\end{enumerate}
\end{theorem}
\end{frame}

\begin{frame}
\addtocounter{theorem}{-1}
\begin{theorem}[Continued]
\begin{enumerate}
\item[($N_{38}$)] $A$ is \defword{inverse-positive}; that is, $A^{-1}$ exists and
\[
A^{-1}\geq 0
\]
\item[($N_{39}$)] $A$ is \defword{monotone}; that is,
\[
Ax\geq 0\Rightarrow x\geq 0\quad\textrm{for all }x\in\IR^n
\]
\item[($N_{40}$)] There exists inverse-positive matrices $B_1$ and $B_2$ such that
\[
B_1\leq A\leq B_2
\]
\item[($N_{41}$)] There exists an inverse-positive matrix $B\geq A$ such that $I-B^{-1}A$ is convergent
\item[($N_{42}$)] There exists an inverse-positive matrix $B\geq A$ and $A$ satisfies ($I_{27}$), ($I_{28}$) and ($I_{29}$)
\end{enumerate}
\end{theorem}
\end{frame}

\begin{frame}
\addtocounter{theorem}{-1}
\begin{theorem}[Continued]
\begin{enumerate}
\item[($N_{43}$)] There exists an inverse-positive matrix $B\geq A$ and a nonsingular M-matrix $C$ such that
\[
A=BC
\]
\item[($N_{44}$)] There exists an inverse-positive matrix $B$ and a nonsingular M-matrix $C$ such that
\[
A=BC
\]
\item[($N_{45}$)] $A$ has a \defword{convergent regular splitting}; that is, $A$ has a representation 
\[
A=M-N,\quad M^{-1}\geq 0,\quad N\geq 0
\]
where $M^{-1}N$ is convergent
\end{enumerate}
\end{theorem}
\end{frame}

\begin{frame}
\addtocounter{theorem}{-1}
\begin{theorem}[Continued]
\begin{enumerate}
\item[($N_{46}$)] $A$ has a \defword{convergent weak regular splitting}; that is, $A$ has a representation
\[
A=M-N,\quad M^{-1}\geq 0,\quad M^{-1}N\geq 0
\]
where $M^{-1}N$ is convergent
\item[($O_{47}$)] Each weak regular splitting of $A$ is convergent
\item[($P_{48}$)] Every regular splitting of $A$ is convergent
\item[($Q_{49}$)] For each $\by\geq\b0$ the set
\[
S_\by=\{\bx\geq\b0: A^T\bx\leq\by\}
\]
is bounded and $A$ is nonsingular
\item[($Q_{50}$)] $S_\b0=\{\b0\}$; that is, the inequalities $A^bx\leq\b0$ and $\bx\geq\b0$ have only the trivial solution $\bx=\b0$ and $A$ is nonsingular
\end{enumerate}
\end{theorem}
\end{frame}


%%%%%%%%%%%%%%%%%%
%%%%%%%%%%%%%%%%%%
%%%%%%%%%%%%%%%%%%
%%%%%%%%%%%%%%%%%%
\begin{frame}[allowframebreaks]
    \frametitle{References}
    \bibliographystyle{amsalpha}
    \bibliography{MATH-4370-7370-lecture-notes.bib}
\end{frame}
    

\end{document}