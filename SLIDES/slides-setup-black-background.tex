\usetheme{default}
% Slide setup, colour independent

\usepackage{amsmath,amssymb,amsthm}
\usepackage{colortbl}
\usepackage{bm}
\usepackage{xcolor}
\usepackage{dsfont}
\usepackage{setspace}
%\usepackage{subfigure}
% To use \ding{234} and the like
\usepackage{pifont}
% To cross reference between slide files
\usepackage{zref-xr,zref-user}
% Use something like
% \zexternaldocument{fileI}
% in the tex files. And cite using \zref instead of \ref

% Fields and the like
\def\IC{\mathbb{C}}
\def\IF{\mathbb{F}}
\def\II{\mathbb{I}}
\def\IJ{\mathbb{J}}
\def\IM{\mathbb{M}}
\def\IN{\mathbb{N}}
\def\IP{\mathbb{P}}
\def\IR{\mathbb{R}}
\def\IZ{\mathbb{Z}}
\def\11{\mathds{1}}


% Bold lowercase
\def\ba{\mathbf{a}}
\def\bb{\mathbf{b}}
\def\bc{\mathbf{c}}
\def\bd{\mathbf{d}}
\def\be{\mathbf{e}}
\def\bf{\mathbf{f}}
\def\bh{\mathbf{h}}
\def\bi{\mathbf{i}}
\def\bj{\mathbf{j}}
\def\bk{\mathbf{k}}
\def\bn{\mathbf{n}}
\def\bp{\mathbf{p}}
\def\br{\mathbf{r}}
\def\bs{\mathbf{s}}
\def\bu{\mathbf{u}}
\def\bv{\mathbf{v}}
\def\bw{\mathbf{w}}
\def\bx{\mathbf{x}}
\def\by{\mathbf{y}}
\def\bz{\mathbf{z}}

% Bold capitals
\def\bB{\mathbf{B}}
\def\bD{\mathbf{D}}
\def\bF{\mathbf{F}}
\def\bG{\mathbf{G}}
\def\bI{\mathbf{I}}
\def\bL{\mathbf{L}}
\def\bN{\mathbf{N}}
\def\bP{\mathbf{P}}
\def\bR{\mathbf{R}}
\def\bS{\mathbf{S}}
\def\bT{\mathbf{T}}
\def\bX{\mathbf{X}}

% Bold numbers
\def\b0{\mathbf{0}}

% Bold greek
\bmdefine{\bmu}{\bm{\mu}}
\def\bphi{\bm{\phi}}
\def\bvarphi{\bm{\varphi}}

% Bold red sentence
\def\boldred#1{{\color{red}\textbf{#1}}}
\def\defword#1{{\color{orange}\textbf{#1}}}

% Caligraphic letters
\def\A{\mathcal{A}}
\def\B{\mathcal{B}}
\def\C{\mathcal{C}}
\def\D{\mathcal{D}}
\def\E{\mathcal{E}}
\def\F{\mathcal{F}}
\def\G{\mathcal{G}}
\def\H{\mathcal{H}}
\def\I{\mathcal{I}}
\def\L{\mathcal{L}}
\def\M{\mathcal{M}}
\def\N{\mathcal{N}}
\def\P{\mathcal{P}}
\def\R{\mathcal{R}}
\def\S{\mathcal{S}}
\def\T{\mathcal{T}}
\def\U{\mathcal{U}}
\def\V{\mathcal{V}}

% tt font for code
\def\code#1{{\tt #1}}

% i.e., e.g.
\def\eg{\emph{e.g.}}
\def\ie{\emph{i.e.}}


% Operators and special symbols
\def\nbOne{{\mathchoice {\rm 1\mskip-4mu l} {\rm 1\mskip-4mu l}
{\rm 1\mskip-4.5mu l} {\rm 1\mskip-5mu l}}}
\def\cov{\ensuremath{\mathsf{cov}}}
\def\Var{\ensuremath{\mathsf{Var}\ }}
\def\Im{\textrm{Im}\;}
\def\Re{\textrm{Re}\;}
\def\det{\ensuremath{\mathsf{det}}}
\def\diag{\ensuremath{\mathsf{diag}}}
\def\nullspace{\ensuremath{\mathsf{null}}}
\def\nullity{\ensuremath{\mathsf{nullity}}}
\def\rank{\ensuremath{\mathsf{rank}}}
\def\range{\ensuremath{\mathsf{range}}}
\def\sgn{\ensuremath{\mathsf{sgn}}}
\def\Span{\ensuremath{\mathsf{span}}}
\def\tr{\ensuremath{\mathsf{tr}}}
\def\imply{$\Rightarrow$}
\def\restrictTo#1#2{\left.#1\right|_{#2}}
\newcommand{\parallelsum}{\mathbin{\!/\mkern-5mu/\!}}
\def\dsum{\mathop{\displaystyle \sum }}%
\def\dind#1#2{_{\substack{#1\\ #2}}}

\DeclareMathOperator{\GL}{GL}
\DeclareMathOperator{\Rel}{Re}
\def\Nt#1{\left|\!\left|\!\left|#1\right|\!\right|\!\right|}
\newcommand{\tripbar}{|\! |\! |}



% The beamer bullet (in base colour)
\def\bbullet{\leavevmode\usebeamertemplate{itemize item}\ }

% Theorems and the like
\newtheorem{proposition}[theorem]{Proposition}
\newtheorem{property}[theorem]{Property}
\newtheorem{importantproperty}[theorem]{Property}
\newtheorem{importanttheorem}[theorem]{Theorem}
%\newtheorem{lemma}[theorem]{Lemma}
%\newtheorem{corollary}[theorem]{Corollary}
\newtheorem{remark}[theorem]{Remark}
\setbeamertemplate{theorems}[numbered]
%\setbeamertemplate{theorems}[ams style]

%
%\usecolortheme{orchid}
%\usecolortheme{orchid}

\def\red{\color[rgb]{1,0,0}}
\def\blue{\color[rgb]{0,0,1}}
\def\green{\color[rgb]{0,1,0}}


% Get rid of navigation stuff
\setbeamertemplate{navigation symbols}{}

% Set footline/header line
\setbeamertemplate{footline}
{%
\quad p. \insertpagenumber \quad--\quad \insertsection\vskip2pt
}
% \setbeamertemplate{headline}
% {%
% \quad\insertsection\hfill p. \insertpagenumber\quad\mbox{}\vskip2pt
% }


\makeatletter
\newlength\beamerleftmargin
\setlength\beamerleftmargin{\Gm@lmargin}
\makeatother

% Colours for special pages
\def\extraContent{yellow!20}


%%%%%%%%%%%%%%%%%
\usepackage{tikz}
\usetikzlibrary{shapes,arrows}
\usetikzlibrary{positioning}
\usetikzlibrary{shapes.symbols,shapes.callouts,patterns}
\usetikzlibrary{calc,fit}
\usetikzlibrary{backgrounds}
\usetikzlibrary{decorations.pathmorphing,fit,petri}
\usetikzlibrary{automata}
\usetikzlibrary{fadings}
\usetikzlibrary{patterns,hobby}

\usetikzlibrary{backgrounds,fit,petri}


\usepackage{pgfplots}
\pgfplotsset{compat=1.6}
\pgfplotsset{ticks=none}

\usetikzlibrary{decorations.markings}
\usetikzlibrary{arrows.meta}
\tikzset{>=stealth}

% For tikz
\usetikzlibrary{shapes,arrows}
\usetikzlibrary{positioning}
\tikzstyle{cloud} = [draw, ellipse,fill=red!20, node distance=0.87cm,
minimum height=2em]
\tikzstyle{line} = [draw, -latex']


%%% For max frame images
\newenvironment{changemargin}[2]{%
\begin{list}{}{%
\setlength{\topsep}{0pt}%
\setlength{\leftmargin}{#1}%
\setlength{\rightmargin}{#2}%
\setlength{\listparindent}{\parindent}%
\setlength{\itemindent}{\parindent}%
\setlength{\parsep}{\parskip}%
}%
\item[]}{\end{list}}


% Make one image take up the entire slide content area in beamer,.:
% centered/centred full-screen image, with title:
% This uses the whole screen except for the 1cm border around it
% all. 128x96mm
\newcommand{\titledFrameImage}[2]{
\begin{frame}{#1}
%\begin{changemargin}{-1cm}{-1cm}
\begin{center}
\includegraphics[width=108mm,height=\textheight,keepaspectratio]{#2}
\end{center}
%\end{changemargin}
\end{frame}
}

% Make one image take up the entire slide content area in beamer.:
% centered/centred full-screen image, no title:
% This uses the whole screen except for the 1cm border around it
% all. 128x96mm
\newcommand{\plainFrameImage}[1]{
\begin{frame}[plain]
%\begin{changemargin}{-1cm}{-1cm}
\begin{center}
\includegraphics[width=108mm,height=76mm,keepaspectratio]{#1}
\end{center}
%\end{changemargin}
\end{frame}
}

% Make one image take up the entire slide area, including borders, in beamer.:
% centered/centred full-screen image, no title:
% This uses the entire whole screen
\newcommand{\maxFrameImage}[1]{
\begin{frame}[plain]
\begin{changemargin}{-1cm}{-1cm}
\begin{center}
\includegraphics[width=\paperwidth,height=\paperheight,keepaspectratio]
{#1}
\end{center}
\end{changemargin}
\end{frame}
}

% This uses the entire whole screen (to include in frame)
\newcommand{\maxFrameImageNoFrame}[1]{
\begin{changemargin}{-1cm}{-1cm}
\begin{center}
\includegraphics[width=\paperwidth,height=0.99\paperheight,keepaspectratio]
{#1}
\end{center}
\end{changemargin}
}

% Make one image take up the entire slide area, including borders, in beamer.:
% centered/centred full-screen image, no title:
% This uses the entire whole screen
\newcommand{\maxFrameImageColor}[2]{
\begin{frame}[plain]
\setbeamercolor{normal text}{bg=#2!20}
\begin{changemargin}{-1cm}{-1cm}
\begin{center}
\includegraphics[width=\paperwidth,height=\paperheight,keepaspectratio]
{#1}
\end{center}
\end{changemargin}
\end{frame}
}


\usepackage{tikz}
\usetikzlibrary{patterns,hobby}
\usepackage{pgfplots}
\pgfplotsset{compat=1.6}
\pgfplotsset{ticks=none}

\usetikzlibrary{backgrounds}
\usetikzlibrary{decorations.markings}
\usetikzlibrary{arrows.meta}
\tikzset{>=stealth}

\tikzset{
  clockwise arrows/.style={
    postaction={
      decorate,
      decoration={
        markings,
        mark=between positions 0.1 and 0.9 step 40pt with {\arrow{>}},
   }}}}


   %%%%%%%%%%%
% To have links to parts in the outline
\makeatletter
\AtBeginPart{%
  \addtocontents{toc}{\protect\beamer@partintoc{\the\c@part}{\beamer@partnameshort}{\the\c@page}}%
}
%% number, shortname, page.
\providecommand\beamer@partintoc[3]{%
  \ifnum\c@tocdepth=-1\relax
    % requesting onlyparts.
    \makebox[6em]{Part #1:} \textcolor{green!30!blue}{\hyperlink{#2}{#2}}
    \par
  \fi
}
\define@key{beamertoc}{onlyparts}[]{%
  \c@tocdepth=-1\relax
}
\makeatother%

\newcommand{\nameofthepart}{}
\newcommand{\nupart}[1]%
    {   \part{#1}%
        \renewcommand{\nameofthepart}{#1}%
        {
          \setbeamercolor{background canvas}{bg=orange!50}
          \begin{frame}{#1}%\partpage 
          \hypertarget{\nameofthepart}{}\tableofcontents%
          \end{frame}
        }
    }


%% Listings
\usepackage{listings}
\definecolor{mygreen}{rgb}{0,0.6,0}
\definecolor{mygray}{rgb}{0.5,0.5,0.5}
\definecolor{mymauve}{rgb}{0.58,0,0.82}
\definecolor{mygold}{rgb}{1,0.843,0}
\definecolor{myblue}{rgb}{0.537,0.812,0.941}

\definecolor{lgreen}{rgb}{0.6,0.9,.6}
\definecolor{lred}{rgb}{1,0.5,.5}

\lstloadlanguages{R}
\lstset{ %
  language=R,
  backgroundcolor=\color{black!95},   % choose the background color
  basicstyle=\footnotesize\ttfamily,        % size of fonts used for the code
  breaklines=true,                 % automatic line breaking only at whitespace
  captionpos=b,                    % sets the caption-position to bottom
  commentstyle=\color{mygreen},    % comment style
  escapeinside={\%*}{*)},          % if you want to add LaTeX within your code
  keywordstyle=\color{myblue},       % keyword style
  stringstyle=\color{mygold},     % string literal style
  keepspaces=true,
  columns=fullflexible,
  tabsize=4,
}
% Could also do (in lstset)
% basicstyle==\fontfamily{pcr}\footnotesize


%%%%%%% 
%% Definitions in yellow boxes
\usepackage{etoolbox}
\setbeamercolor{block title}{use=structure,fg=structure.fg,bg=structure.fg!05!bg}
\setbeamercolor{block body}{parent=normal text,use=block title,bg=block title.bg!20!bg}

\BeforeBeginEnvironment{definition}{%
	\setbeamercolor{block title}{fg=black,bg=yellow!20!white}
	\setbeamercolor{block body}{fg=black, bg=yellow!05!white}
}
\AfterEndEnvironment{definition}{
	\setbeamercolor{block title}{use=structure,fg=structure.fg,bg=structure.fg!20!bg}
	\setbeamercolor{block body}{parent=normal text,use=block title,bg=block title.bg!50!bg, fg=black}
}
\BeforeBeginEnvironment{importanttheorem}{%
	\setbeamercolor{block title}{fg=black,bg=red!20!white}
	\setbeamercolor{block body}{fg=black, bg=red!05!white}
}
\AfterEndEnvironment{importanttheorem}{
	\setbeamercolor{block title}{use=structure,fg=structure.fg,bg=structure.fg!20!bg}
	\setbeamercolor{block body}{parent=normal text,use=block title,bg=block title.bg!50!bg, fg=black}
}
\BeforeBeginEnvironment{theorem}{%
	\setbeamercolor{block title}{fg=white,bg=red!30!black}
	\setbeamercolor{block body}{fg=white, bg=red!10!black}
}
\AfterEndEnvironment{theorem}{
	\setbeamercolor{block title}{use=structure,fg=structure.fg,bg=structure.fg!20!bg}
	\setbeamercolor{block body}{parent=normal text,use=block title,bg=block title.bg!50!bg, fg=black}
}
\BeforeBeginEnvironment{importantproperty}{%
	\setbeamercolor{block title}{fg=black,bg=red!50!white}
	\setbeamercolor{block body}{fg=black, bg=red!30!white}
}
\AfterEndEnvironment{importantproperty}{
	\setbeamercolor{block title}{use=structure,fg=structure.fg,bg=structure.fg!20!bg}
	\setbeamercolor{block body}{parent=normal text,use=block title,bg=block title.bg!50!bg, fg=black}
}


\tikzstyle{cloud} = [draw, 
ellipse,
fill=red!20, 
node distance=0.87cm,
minimum height=2em]
\tikzstyle{line} = [draw, 
-latex', 
color=yellow]


% Beginning of a section
% \AtBeginSection[]{
% 	{
% 		\setbeamercolor{background canvas}{bg=orange!10}
% 		\begin{frame}[noframenumbering,plain]
% 			\framesubtitle{\nameofthepart Chapter \insertromanpartnumber \ -- \iteminsert{\insertpart}}
% 			\tableofcontents[currentsection,currentsubsection]
% 		\end{frame}
% 	\addtocounter{page}{-1}
% 	%\addtocounter{framenumber}{-1} 
% 	}
% }


%%% SLIDES COLOURING

%\usecolortheme{owl}

\setbeamerfont{frametitle}{series=\bfseries}
\setbeamercolor{frametitle}{fg=black!05,bg=black}

\setbeamerfont{framesubtitle}{size=\normalfont\tiny}
\setbeamercolor{framesubtitle}{fg=black!05}

\setbeamercolor{background canvas}{bg=black}
\setbeamercolor{normal text}{fg=black!10}



\definecolor{bottomcolour}{rgb}{0.32,0.3,0.38}
\definecolor{middlecolour}{rgb}{0.08,0.08,0.16}
\definecolor{mycolor}{rgb}{0.4,0.4, 0.4}
% Beginning of a section
\AtBeginSection[]{
	{
		%\setbeamercolor{background canvas}[vertical shading][top=bottomcolour, middle=middlecolour, bottom=black]
		\setbeamertemplate{background canvas}[vertical shading][bottom=bottomcolour,top=black!20]
    % \setbeamertemplate{background canvas}{
    %   \begin{tikzpicture}%[remember picture,overlay]
    %     \shade[top color=yellow!75!green!33,
    %     bottom color=blue!66!green!33,
    %     middle color=blue!6!green!33]
    %   \end{tikzpicture}
    % }
    \begin{frame}[noframenumbering,plain]
			\framesubtitle{\nameofthepart Chapter \insertromanpartnumber \ -- \iteminsert{\insertpart}}
			\tableofcontents[currentsection,currentsubsection]
		\end{frame}
	\addtocounter{page}{-1}
	}
}
% Beginning of a section
\AtBeginSubsection[]{
	{
		%\setbeamercolor{background canvas}[vertical shading][top=bottomcolour, middle=middlecolour, bottom=black]
		%\setbeamertemplate{background canvas}[vertical shading][bottom=bottomcolour,top=black!20]
    \setbeamertemplate{background canvas}{
      \begin{tikzpicture}%[remember picture,overlay]
        \shade[top color=yellow!75!green!33,
        bottom color=blue!66!green!33,
        middle color=blue!6!green!33]
      \end{tikzpicture}
    }
    \begin{frame}[noframenumbering,plain]
			\framesubtitle{\nameofthepart Chapter \insertromanpartnumber \ -- \iteminsert{\insertpart}}
			\tableofcontents[currentsection,currentsubsection]
		\end{frame}
	\addtocounter{page}{-1}
	}
}

% Colours for special pages
\def\extraContent{yellow!20}

%% Allow to change slide colour
%% From: https://tex.stackexchange.com/questions/8043/change-the-background-color-of-a-frame-in-beamer
\defbeamertemplate*{background canvas}{mydefault}{%
  \ifbeamercolorempty[bg]{background canvas}{}{\color{bg}\vrule width\paperwidth height\paperheight}% copied beamer default here
}
\defbeamertemplate*{background canvas}{bg}{%
  \color{lightgray!20}\vrule width\paperwidth height\paperheight% added bg color
}
\BeforeBeginEnvironment{frame}{%
  \setbeamertemplate{background canvas}[mydefault]%
}
\makeatletter
\define@key{beamerframe}{bg}[true]{%
  \setbeamertemplate{background canvas}[bg]%
}
\makeatother
% Use with
%\begin{frame}
% \frametitle{Normal}
%\end{frame} 
%\begin{frame}[bg]
% \frametitle{With bg}
%\end{frame}


%% Vertical alignment on pages
%% From: https://tex.stackexchange.com/questions/148365/how-do-i-ask-beamer-to-exactly-fill-up-a-slide
%% Turn on with
%% \stretchon
%% (outside slide), and off with
%% \stretchoff
% \def\itemsymbol{$\blacktriangleright$}
% \let\svpar\par
% \let\svitemize\itemize
% \let\svenditemize\enditemize
% \let\svitem\item
% \let\svcenter\center
% \let\svendcenter\endcenter
% \let\svcolumn\column
% \let\svendcolumn\endcolumn
% \def\newitem{\renewcommand\item[1][\itemsymbol]{\vfill\svitem[##1]}}%
% \def\newpar{\def\par{\svpar\vfill}}%
% \newcommand\stretchon{%
%   \newpar%
%   \renewcommand\item[1][\itemsymbol]{\svitem[##1]\newitem}%
%   \renewenvironment{itemize}%
%     {\svitemize}{\svenditemize\newpar\par}%
%   \renewenvironment{center}%
%     {\svcenter\newpar}{\svendcenter\newpar}%
%   \renewenvironment{column}[2]%
%     {\svcolumn{##1}\setlength{\parskip}{\columnskip}##2}%
%     {\svendcolumn\vspace{\columnskip}}%
% }
% \newcommand\stretchoff{%
%   \let\par\svpar%
%   \let\item\svitem%
%   \let\itemize\svitemize%
%   \let\enditemize\svenditemize%
%   \let\center\svcenter%
%   \let\endcenter\svendcenter%
%   \let\column\svcolumn%
%   \let\endcolumn\svendcolumn%
% }
% \newlength\columnskip
% \columnskip 0pt
