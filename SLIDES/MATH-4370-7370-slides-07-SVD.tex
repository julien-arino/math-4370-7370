\documentclass[aspectratio=169]{beamer}

%\documentclass[handout]{beamer}
%% To make 4 per page
%\usepackage{pgfpages}
%\mode<handout>{\setbeamercolor{background canvas}{bg=white}}
%\pgfpagesuselayout{4 on 1}[letterpaper,landscape]%,border shrink=5mm]

\input{slides-setup-white-background.tex}

% To have theorems and everything derived prefixed by the slide set
% number
\renewcommand{\thetheorem}{5.\arabic{theorem}}



\title{The Singular Value Decomposition}
\author{Julien Arino}
\date{Fall 2025}

%%%%%%%%%%%%%%%%%%%%%
%%%%%%%%%%%%%%%%%%%%%
%%%%%%%%%%%%%%%%%%%%%
%%%%%%%%%%%%%%%%%%%%%
%%%%%%%%%%%%%%%%%%%%%
%%%%%%%%%%%%%%%%%%%%%
%%%%%%%%%%%%%%%%%%%%%
%%%%%%%%%%%%%%%%%%%%%
\begin{document}

% The title page
\begin{frame}[noframenumbering,plain]
  \begin{tikzpicture}[remember picture,overlay]
    \node[above right,inner sep=0pt] at (current page.south west)
    {
        \includegraphics[width=\paperwidth]{FIGS/title-page-picture.png}
    };
\end{tikzpicture}
	\titlepage
\end{frame}
\addtocounter{page}{-1}
  
  
\begin{frame}{Outline}
	  \tableofcontents[hideallsubsections]
\end{frame}
\addtocounter{page}{-1}

%%%%%%%%%%%%%%%%%%%%%%%%%%%
%%%%%%%%%%%%%%%%%%%%%%%%%%%
%%%%%%%%%%%%%%%%%%%%%%%%%%%
%%%%%%%%%%%%%%%%%%%%%%%%%%%
\section{Singular values and the Singular value decomposition}
\label{sec:SVD}


\begin{frame}
	\begin{definition}
		Let $A$ be a Hermitian matrix in $\M_n$. We say that $A$ is \defword{positive definite}\index{matrix!positive definite} if for all $\b0\neq\bx\in \IC^n$, $\bx^{*}A\bx>0$. We say that $A$ is \defword{positive semidefinite} if for all $\bx\in \IC^n$, $\bx\neq\b0$, $\bx^* A\bx\geq 0$
	\end{definition}
	\vfill
	\begin{theorem}
		Let $A\in \M_n$ be a Hermitian matrix. Then 
		\begin{enumerate}
			\item for all $\bx\in \IC^*$, $\bx^*A\bx\in\IR$ 
			\item $\sigma(A)\subset \IR$
			\item $S^*AS$ is Hermitian for any $S\in \M_n$
		\end{enumerate}
	\end{theorem}
	\vfill
	\begin{theorem}
		Each eigenvalue of a positive definite matrix (respectively positive semidefinite matrix) is positive (respectively nonnegative)
	\end{theorem}
\end{frame}


\begin{frame}
	\begin{proposition}
		Let $A$ be a positive semidefinite (respectively positive definite) matrix. Then $\tr (A)$, $\det(A)$, the principal minors of $A$ are all nonnegative (respectively positive). Also, $\tr(A)=0$ if and only if $A=0$
	\end{proposition}
	
	\begin{theorem}
		Let $A\in \M_n$ be a positive semidefinite matrix and $\bx\in \IC^n$. Then 
		\[
		\bx^*A\bx= 0 \iff A\bx=\b0
		\]
	\end{theorem}
	
	\begin{corollary}
		Let $A\in \M_n$ be a positive semidefinite matrix. Then $A$ is positive definite if and only if $A$ is nonsingular
	\end{corollary}
\end{frame}


\begin{frame}
	\begin{theorem}[Somewhat unrelated]
		Let $B\in \M_n$ be a Hermitian matrix, $\by \in \IC^n$, and $a\in \IR$. Let 
		\[
		A= \begin{pmatrix}
			B& \by\\
			\by^*&a
		\end{pmatrix}\in \M_{n+1}
		\]
		Then 
		\[
		\lambda_1(A) \leq \lambda_1(B)\leq \lambda_2(A)\leq \dots \leq 
		\lambda_n(A)\leq \lambda_n(B) \leq \lambda_{n+1}(A)
		\]
	\end{theorem}
\end{frame}


\begin{frame}
	\begin{definition}
		The singular values of a matrix $A$ are the (nonnegative) square roots of the eigenvalues of $A^*A$
	\end{definition}
	\vfill
	\begin{remark}
		$A^*A$ is positive semidefinite
	\end{remark}
\end{frame}


\begin{frame}
	\begin{theorem}[Zhang]\label{Theorem:ZhangSV}
		Let $A\in \M_{mn}$ with nonzero singular values $\sigma_1, \dots, \sigma_r$. Then there exists $U\in \M_n$ and $V \in \M_n$ unitary such that 
		\[A= U \begin{pmatrix}
			D_r& 0\\
			0& 0
		\end{pmatrix}V,\]
		where $\begin{pmatrix}
			D_r& 0\\
			0& 0
		\end{pmatrix}\in M_{mn}$ and $D_r= \diag(\sigma_1, \dots, \sigma_r)$
	\end{theorem}
\end{frame}



\begin{frame}
	\begin{theorem}[H \& J] Let $A\in \M_{nm}$ , $q= \min\{m,n\}$. Assume that the rank of $A$ is n. Then 
		\begin{enumerate}
			\item $\exists V \in M_{n}$ and $W \in \M_{m}$ unitary matrices and $\Sigma_q\in \M_q= \diag(\sigma_1, \dots, \sigma_q)$ s.t. 
			\[\sigma_1\geq \sigma_2 \geq \dots \geq \sigma_r > 0= \sigma_{r+1} = \dots = \sigma_q\]
			and
			\[A\Sigma W\]
			where 
			\[\Sigma= \begin{cases}
				\Sigma_1, & m=n\\
				\begin{pmatrix}
					\Sigma_q&
					0
				\end{pmatrix}\in \M_{nm},& m>n\\
				\begin{pmatrix}
					\Sigma_q\\
					0
				\end{pmatrix}\in \M_{nm},&  n>m
			\end{cases}
			\]
			\item The parameters $\sigma_1, \dots, \sigma_r$ are the positive square roots of the decreasingly ordered eigenvalues of $A^*A$
		\end{enumerate}
		
	\end{theorem}
\end{frame}


\begin{frame}
	\begin{remark}
		Let $A\in \M_{mn}$. Then $A, \overline{A}, A^T$, and $A^*$ have the same singular values
	\end{remark}
	\vfill
	\begin{remark}
		Let $A\in \M_n$ with singular values $\sigma_1, \dots, \sigma_n$, then 
		\[
		\sigma_1\ldots \sigma_n = \det(A)
		\]
		and 
		\[
		\sigma_1^2+ \ldots+ \sigma_n^2= \tr(A^*A)
		\]
	\end{remark}
\end{frame}


\begin{frame}
	\begin{theorem}
		Let $A\in \M_{nm}$, $q= \min{m,n}$, and $\sigma_1\geq \dots \geq \sigma_q$ nonincresingly ordered singular values of $A$. Define 
		\[\mathcal{A}= \begin{pmatrix}
			0& A\\
			A^*& 0
		\end{pmatrix}\]
		to be a Hermitian matrix. Then the ordered eigenvalues of $\mathcal{A}$ are 
		\[
		-\sigma_1 \leq \dots \leq -\sigma_q \leq \underbrace{0= \dots = 0}{|n-m|} \leq \sigma_q\leq \dots \leq \sigma_1
		\]
	\end{theorem}
\end{frame}



\begin{frame}
	\begin{theorem}[An interlacing result] Let $A\in \M_{nm}$, $q= \min\{m,n\}$ and $\hat{A}$ be the matrix obtained from $A$
		by deleting one row and one column. Let $\sigma_1 \geq \dots \geq \sigma_q$ and $\hat{\sigma}_1\geq \dots \geq \hat{\sigma}_q$ be the nonsingular ordered singular values of $A$ and $\hat{A}$, respectively, where 
		$\hat{\sigma}_q=0$ if $n\geq m$ and a column is deleted or if $n\geq m$ and a row is deleted. Then
		\[\sigma_1 \geq \hat{\sigma_1} \geq \sigma_2 \geq \hat{\sigma_2} \geq \dots \sigma_q \geq \hat{\sigma_q}.\]
	\end{theorem}
	
	\begin{theorem}[von Neumann] Let $A, \, B\in \M_{mn}$, $q= \min\{m, n\}$, $\sigma_1(A)\geq \dots \geq \sigma_q(A)$ and $\sigma_1(B)\geq \dots \geq \sigma_q(B)$ the non-increasingly singular values of $A$ and $B$, respectively. Then 
		\[\Rel\tr(AB^*)\leq \sum\limits_{i=1}^q \sigma_i(A) \sigma_i(B).\]
	\end{theorem}
\end{frame}


\begin{frame}
	\begin{theorem}
		Let $A\in \M_{nm}$, $q= \min{m,n}$, and $\sigma_1\geq \dots \geq \sigma_q$ nonincreasingly ordered singular values of $A$, and $\alpha= \{1, \dots, q\}$. Then 
		\[\Rel\tr(A)\leq \sum\limits_{i=1}^q\sigma_i\]
		with equality if and only if $A[\alpha]$ (principal leading submatrix of $A$) is positive semidefinite and $A$ has no nonzero entries outside $A[\alpha]$. 
	\end{theorem}
\end{frame}

%%%%%%%%%%%%%%%%%%%%%%%%%%%
%%%%%%%%%%%%%%%%%%%%%%%%%%%
%%%%%%%%%%%%%%%%%%%%%%%%%%%
%%%%%%%%%%%%%%%%%%%%%%%%%%%
\section{Properties of Singular Values}


\begin{frame}
	\begin{itemize}
		\item Let $A\in\M_2$
		\[\sigma_1, \sigma_2= \dfrac{1}{2}\left((\tr A^*A) \mp \sqrt{(\tr A^* A)^2-4|\det A|^2} \right)\]
		\item The nilpotent matrix 
		\[A= \begin{pmatrix}
			0& a_{12}& \\
			& \ddots& \\
			& & & a_{n-1, m}\\
			& & &0
		\end{pmatrix}\]
		has singular values $0, |a_{12}|, \dots, |a_{n-1,n}| $.
	\end{itemize}
\end{frame}

\begin{frame}
	\begin{theorem}
		Let $A_1, A_2, \dots \in \M_{nm}$ given (infinite) sequence with $\lim_{k\to \infty} A_k=A$ (entrywise). Let $q= \min (m,n)$. Let $\sigma_1 (A)\geq \dots \geq \sigma_q(A)$ and $\sigma_1 (A_k)\geq \dots \geq \sigma_q(A_k)$ be the non-increasinly ordered singular values of $A$ and $A_k$, respectively (for all $k$). Then 
		\[
		\lim_{k\to \infty}\sigma_i(A_k)= \sigma_i(A)
		\]
	\end{theorem}
\end{frame}

\begin{frame}
	\begin{theorem}
		Let $A\in \M_n$ where $n= \rank\ A$
		\begin{enumerate}
			\item $A= A^T$ if and only if there exists $U \in \M_n$ unitary and a nonegative diagonal matrix $\Sigma$ such that $A= U \Sigma U^T$. Then the diagonal entries of $\Sigma$ are the singular values of $A$
			\item If $A=-A^T$, then $n$ is even and there exists $U \in \M_n$ unitary and positive real scalars $s_1, \dots , s_{r/2}$ such that 
			\[
			U \left( \begin{pmatrix}
				0& s_1\\
				-s_1& 0
			\end{pmatrix} \oplus \dots \oplus 
			\begin{pmatrix}
				0& s_{r/2}\\
				-s_{r/2}& 0
			\end{pmatrix}\right)U^T
			\]
			The non-zero singular values of $A$ are $s_1,s_1,  \dots, s_{r/2}, s_{r/2}$.
			Conversely, any matrix of the above form is skew-symetric
		\end{enumerate}
	\end{theorem}
\end{frame}




%%%%%%%%%%%%%%%%%%%%%%%%%%%
%%%%%%%%%%%%%%%%%%%%%%%%%%%
%%%%%%%%%%%%%%%%%%%%%%%%%%%
%%%%%%%%%%%%%%%%%%%%%%%%%%%
\section{Matrix norms and Singular values}
\label{sec:matrixNorms_SVD}

\begin{frame}
Let $V = \M_{mn}(\IC)$ with Frobenius inner product 
\[
    \langle A, \, B \rangle_{F}= \tr(B^*A)
\]
The norm derived from the Frobenius inner product is 
\[\|A\|_2= (\tr(A^*A))^{1/2}\]
is the $\ell$-2 norm (or Frobenius norm)
\end{frame}

\begin{frame}
The spectral norm $\Nt{\cdot}$ defined on $\M_n$ by 
\[\tripbar A\tripbar_2= \sigma_1(A),\]
where $\sigma_1(A)$ is the largest singular value of $A$ is induced by the $\ell$-2 norm on $\IC^n$. \\
Inded, from the singular value decomposition theorem, let 
\[A= V \Sigma W^*\]
be a singular value decomposition of $A$, where $V, \, W$ unitary, $\Sigma= \sigma(\sigma_1, \dots, \sigma_n)$ and $\sigma_1 \geq \dots \geq \sigma_n\geq 0$ are the non-increasingly ordered singular values of $A$
\end{frame}


\begin{frame}
From unitary invariance and monotonicity of the Euclidean norm, we say that 
\begin{align*}
\max \|Ax\|_1&= \max_{\|x\|_1} \|V\Sigma W^*\|_2\\
&= \max_{\|x\|_2} \|\Sigma W^* x\|_2\\
&= \max_{\|Wy\|_2=1}\|\Sigma y\|_2\\
&= \max_{\|y\|_2}\|\Sigma y\|_2\\
& \leq \max_{\|y\|_2}\|\sigma_1 y\|_2\\
&= \sigma_1 \max_{\| y \|_2} \|y\|_2\\
&= \sigma_1
\end{align*}
Since $\| \Sigma y\|_2= \sigma_1$ for $y= e_1$, 
\[\max_{\|x\|_2=1} \|Ax\|_2= \sigma_1(A)\]
\end{frame}


\begin{frame}
We could have used 
\begin{align*}
\max_{\|x\|_2=1}=\|Ax\|_2^2&= \max_{\|x\|_2=1}x^*A^*AX\\
&= \lambda_{max}(A^*A)\\
&= \sigma_1(A)
\end{align*}
\vfill
\begin{proposition}
	For all $U,\, V\in\M_n$ unitary matrices, we have $\tripbar UAV\tripbar_2= \tripbar A\tripbar_2$
\end{proposition}
\vfill

\end{frame}



%%%%%%%%%%%%%%%%%%
%%%%%%%%%%%%%%%%%%
%%%%%%%%%%%%%%%%%%
%%%%%%%%%%%%%%%%%%
\begin{frame}[allowframebreaks]
    \frametitle{References}
    \bibliographystyle{amsalpha}
    \bibliography{MATH-4370-7370-lecture-notes.bib}
\end{frame}
    

\end{document}